\documentclass[12pt,letterpaper]{article}

% to-do list
% ----------

% style notes
% -----------
% - This file generates by Makefile; don't be typing ``pdflatex'' or some
%   bullshit.
% - Line break between sentences to make the git diffs readable.
% - Use \, as a multiply operator.
% - Reserve () for function arguments; use [] or {} for outer shit.
% - Use \sectionname not Section, \figname not Figure, \documentname not Article
%   or Paper or paper.

\include{preamble}
\include{gitstuff}

% packages
\graphicspath{{figures/}}

\newcommand{\gaia}{\project{Gaia}}
\newcommand{\tgas}{\acronym{TGAS}}

\begin{document}\sloppy\sloppypar\raggedbottom\frenchspacing % trust Hogg

\title{OMG who cares if they are bound}

\author{
  Adrian M. Price-Whelan
}

\maketitle

\section{The model}
\label{sec:model}

The \tgas\ catalog provides measurements of sky position, $(\alpha, \delta)$,
parallax, $\pi$, and proper motions, $(\mu_{\alpha^*}, \mu_\delta)$ along with
the covariance matrix, $\mat{C}^{*}$, associated with these data for each star
in the pair.
Radial velocity measurements, $v_r$, with uncertainties $\sigma_{v_r}$ are
provided from high-resolution spectroscopy.
For brevity, we pack the 6D kinematic data into the vector
\begin{eqnarray}
  \vec{y} &=&
      \transpose{\left(
        \begin{array}{c c c c c c}
          \alpha &
          \delta &
          \pi &
          \mu_{\alpha^*} &
          \mu_\delta &
          v_r
        \end{array}
      \right)}
\end{eqnarray}
and define a new $6\times6$ covariance matrix $\mat{C}$ that contains the
astrometric covariance matrix in the upper-left block, the radial-velocity
variance $\sigma_{v_r}^2$ in the bottom-right diagonal element, and all other
matrix elements set to zero.
We additionally assume we have measurements of the masses of the two stars,
$\hat{M}_1$ and $\hat{M}_1$ with uncertainties $\sigma_{M_1}$ and
$\sigma_{M_2}$, respectively.

Our model parameters include, amongst others, the true 6D Cartesian phase-space
coordinates for each star, $\vec{w}_1=(\vec{x}_1, \vec{v}_1)$ and
$\vec{w}_2=(\vec{x}_2, \vec{v}_2)$. These vectors $\vec{w}$ can be transformed to
the data-space---heliocentric sperical coordinates, where $\vec{w}$ becomes
$\vec{w}^*$---where the likelihood evaluation for the kinematic data is a simple
multivariate Gaussian:
\begin{eqnarray}
    p(\vec{y}_1, \vec{y}_2 \given \vec{w}_1, \vec{w}_2, \mat{C}_1, \mat{C}_2)
        &=&
        \norm\left(\vec{y}_1 \given \vec{w}_1^*, \mat{C}_1 \right) \,
        \norm\left(\vec{y}_2 \given \vec{w}_2^*, \mat{C}_2 \right)
\end{eqnarray}
where
\begin{eqnarray}
    \norm\left(\vec{x} \given \vec{\mu}, \mat{\Sigma}\right) &=&
        \left[\det\left(\frac{\mat{\Sigma}^{-1}}{2\pi}\right)\right]^{1/2} \,
          \exp \left[ -\frac{1}{2} \transpose{\left(\vec{x} - \vec{\mu} \right)} \,
          \mat{\Sigma}^{-1} \,
          \left(\vec{x} - \vec{\mu} \right) \right] \quad .
\end{eqnarray}
The likelihood for the mass measurements given the true masses $M_1$ and $M_2$
are also Gaussian:
\begin{eqnarray}
    p(\hat{M}_1 \given M_1, \sigma_{M_1}) &=&
        \norm\left(\hat{M}_1 \given M_1, \sigma_{M_1}^2\right) \\
    p(\hat{M}_2 \given M_2, \sigma_{M_2}) &=&
        \norm\left(\hat{M}_2 \given M_2, \sigma_{M_2}^2\right)
\end{eqnarray}

We consider three possible models for the prior over $(\vec{w}_1, \vec{w}_2)$:
\begin{enumerate}
    \item The stars are a bound, wide-binary system;
    \item The stars are co-moving within some tolerance (velocity dispersion)
    but unbound;
    \item The stars are independently sampled from the field population.
\end{enumerate}
In case 1 (wide-binary) the prior can be expressed in terms of the 6D barycentric phase-space coordinates of the pair, $\vec{w}_{\rm bary}$, and the difference in 6D coordinates, $\Delta\vec{w}$
\begin{eqnarray}
    p_1(\vec{w}_1, \vec{w}_2) &=&
        \int p()
\end{eqnarray}

Combining these precise radial velocities with the \gaia\ astrometry, we can
compare differences between the inferred 6D phase-space properties for the
two stars.
We start by generating posterior samples over the Heliocentric distance, $r$,
tangential velocities, $(v_\alpha, v_\delta)$, and radial velocity, $\hat{v}_r$,
given the observed parallax, $\pi$, and proper motion components,
$(\mu_{\alpha^*}, \mu_\delta)$, and radial velocity, $v_r$.
By defining the vectors
% SMOH: I understand you meant hat v_r as the latent true radial velocity but
% personally think a) it can confuse some people that there are hat quantities
% with no-hat quantities for hat-y vector and thus b) it'd be better to use hat
% quantities for observed variables and no-hat for latent true ones.
% SMOH: v_r is missing in the LHS of eq(3)
\begin{eqnarray}
  \vec{y} &=&
      \transpose{\left(
        \begin{array}{c@{\hspace{1em}} c@{\hspace{1em}} c@{\hspace{1em}} c}
          \pi &
          \mu_{\alpha^*} &
          \mu_\delta &
          v_r
        \end{array}
      \right)}\\
  \hat{\vec{y}} &=&
      \transpose{\left(
        \begin{array}{c@{\hspace{1em}} c@{\hspace{1em}} c@{\hspace{1em}} c}
          r^{-1} &
          r^{-1}\,v_\alpha &
          r^{-1}\,v_\delta &
          \hat{v}_r
        \end{array}
      \right)}
\end{eqnarray}
and by considering the covariance matrix $\mat{C}$ between the observed
quantities (provided by \tgas\ and extended to include the uncorrelated radial
velocity uncertainty), the likelihood can be written
\begin{eqnarray}
p(\pi, \mu_{\alpha^*}, \mu_\delta \given r, v_\alpha, v_\delta) &=&
  \left[\det\left(\frac{\mat{C}^{-1}}{2\pi}\right)\right]^{1/2} \,
    \exp \left[ -\frac{1}{2} \transpose{\left(\vec{y} - \hat{\vec{y}} \right)} \,
    \mat{C}^{-1} \,
    \left(\vec{y} - \hat{\vec{y}} \right) \right] \label{eq:likefn} \quad .
\end{eqnarray}
We adopt a uniform space density prior for the distance and an isotropic
Gaussian for any velocity component, $v$, with a dispersion $\sigma_v=25~\kms$
\begin{eqnarray}
p(r) &=&
  \begin{cases}
    \frac{3}{r_{\rm lim}^3} \, r^2 & \text{if } 0 < r < r_{\rm lim}\\
    0              & \text{otherwise}
  \end{cases}\\
p(v) &=& \frac{1}{\sqrt{2\pi}\,\sigma_v} \,
  \exp\left[-\frac{1}{2} \, \frac{v^2}{\sigma_v^2} \right] \quad .
\end{eqnarray}
For each of the two stars, we use \project{emcee}
to generate posterior samples in $(r, v_\alpha,
v_\delta, \hat{v}_r)$ by running 64 walkers for 4608 steps and discarding the
first 512 steps as the burn-in period.
For each sample, we convert the heliocentric phase-space coordinates into
Galactocentric coordinates assuming the Sun's position and velocity are $x_\odot
= (-8.3, 0, 0)~{\rm kpc}$ and $v_\odot = (10, 244, 7.17)~\kms$.

\section{Methods}
\label{sec:methods}

\section{Data}
\label{sec:data}

%\bibliography{ref}



\end{document}
