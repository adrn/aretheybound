\documentclass[12pt,letterpaper]{article}

% to-do list
% ----------

% style notes
% -----------
% - This file generates by Makefile; don't be typing ``pdflatex'' or some
%   bullshit.
% - Line break between sentences to make the git diffs readable.
% - Use \, as a multiply operator.
% - Reserve () for function arguments; use [] or {} for outer shit.
% - Use \sectionname not Section, \figname not Figure, \documentname not Article
%   or Paper or paper.

% Load common packages

\usepackage{amsmath}
\usepackage{amsfonts}
\usepackage{amssymb}
\usepackage{booktabs}

\usepackage{graphicx}
\usepackage{color}
\usepackage{microtype}  % ALWAYS!

\newcommand{\documentname}{\textsl{Article}}
\newcommand{\sectionname}{Section}
\newcommand{\figname}{Figure}
\newcommand{\eqname}{Equation}
\renewcommand{\tablename}{Table}

% Packages / projects / programming
\newcommand{\package}[1]{\texttt{#1}}
\newcommand{\acronym}[1]{{\small{#1}}}
\newcommand{\github}{\package{GitHub}}
\newcommand{\python}{\package{Python}}

% Missions / surveys
\newcommand{\project}[1]{\textsl{#1}}

% For referee
\newcommand{\changes}[1]{{\color{red} #1}}

% Stats / probability
\newcommand{\given}{\,|\,}
\newcommand{\norm}{\mathcal{N}}

% Maths
\newcommand{\dd}{\mathrm{d}}
\newcommand{\transpose}[1]{{#1}^{\mathsf{T}}}
\newcommand{\inverse}[1]{{#1}^{-1}}
\newcommand{\argmin}{\operatornamewithlimits{argmin}}
\newcommand{\mean}[1]{\left< #1 \right>}
\newcommand{\mat}[1]{\mathbf{#1}}
\renewcommand{\vec}[1]{\bs{#1}}

% Unit shortcuts
\newcommand{\msun}{\ensuremath{\mathrm{M}_\odot}}
\newcommand{\kms}{\ensuremath{\mathrm{km}~\mathrm{s}^{-1}}}
\newcommand{\pc}{\ensuremath{\mathrm{pc}}}
\newcommand{\kpc}{\ensuremath{\mathrm{kpc}}}
\newcommand{\kmskpc}{\ensuremath{\mathrm{km}~\mathrm{s}^{-1}~\mathrm{kpc}^{-1}}}
\newcommand{\masyr}{\ensuremath{\mathrm{mas}~\mathrm{yr}^{-1}}}

% Misc.
\newcommand{\bs}[1]{\boldsymbol{#1}}

% Astronomy
\newcommand{\DM}{{\rm DM}}
\newcommand{\feh}{\ensuremath{{[{\rm Fe}/{\rm H}]}}}
\newcommand{\df}{\acronym{DF}}

% TO DO
\newcommand{\todo}[1]{{\color{red} TODO: #1}}

\include{gitstuff}

% packages
\graphicspath{{figures/}}

\newcommand{\gaia}{\project{Gaia}}
\newcommand{\tgas}{\acronym{TGAS}}

\begin{document}\sloppy\sloppypar\raggedbottom\frenchspacing % trust Hogg

\title{OMG who cares if they are bound}

\author{
  Adrian M. Price-Whelan
}

\maketitle

\section{The model}
\label{sec:model}

The \tgas\ catalog provides measurements of sky position, $(\alpha, \delta)$,
parallax, $\pi$, and proper motions, $(\mu_{\alpha^*}, \mu_\delta)$, along with
the covariance matrix, $\tilde{\mat{C}}$, associated with these data for each
star in the pair.
Each star additionally has a precise radial-velocity measurement, $v_r$, with
uncertainty $\sigma_{v_r}$ derive from high-resolution, high signal-to-noise
spectroscopy.
For brevity in defining probability distributions below, we pack these 6D
kinematic data into vectors
\begin{eqnarray}
  \vec{y} &=&
      \transpose{\left(
        \begin{array}{c c c c c c}
          \alpha &
          \delta &
          \pi &
          \mu_{\alpha^*} &
          \mu_\delta &
          v_r
        \end{array}
      \right)}
\end{eqnarray}
and define a new $6\times6$ covariance matrix $\mat{C}$ for each star that
contains the astrometric covariance matrix in the upper-left block, the
radial-velocity variance $\sigma_{v_r}^2$ in the bottom-right diagonal element,
and all other matrix elements set to zero.
We additionally have measurements of the masses of the two stars, $\hat{M}_1$
and $\hat{M}_1$ with uncertainties $\sigma_{M_1}$ and $\sigma_{M_2}$,
respectively.

Our model parameters include, amongst others, the true 6D Cartesian phase-space
coordinates for each star, $\vec{w}_1=(\vec{x}_1, \vec{v}_1)$ and
$\vec{w}_2=(\vec{x}_2, \vec{v}_2)$.
These vectors $\vec{w}$ can be transformed to the data-space---heliocentric
spherical coordinates, where $\vec{w}$ becomes $\vec{w}^*$---where the
likelihood evaluation for the kinematic data is a multivariate Gaussian:
\begin{eqnarray}
    p(\vec{y}_1, \vec{y}_2 \given \vec{w}_1, \vec{w}_2, \mat{C}_1, \mat{C}_2)
        &=&
        \norm\left(\vec{y}_1 \given \vec{w}_1^*, \mat{C}_1 \right) \,
        \norm\left(\vec{y}_2 \given \vec{w}_2^*, \mat{C}_2 \right)
\end{eqnarray}
with
\begin{eqnarray}
    \norm\left(\vec{x} \given \vec{\mu}, \mat{\Sigma}\right) &=&
        \left[\det\left(\frac{\mat{\Sigma}^{-1}}{2\pi}\right)\right]^{1/2} \,
          \exp \left[ -\frac{1}{2} \transpose{\left(\vec{x} - \vec{\mu} \right)} \,
          \mat{\Sigma}^{-1} \,
          \left(\vec{x} - \vec{\mu} \right) \right] \quad .
\end{eqnarray}
The likelihood for the mass measurements given the true masses $M_1$ and $M_2$
are also Gaussian:
\begin{eqnarray}
    p(\hat{M}_1 \given M_1, \sigma_{M_1}) &=&
        \norm\left(\hat{M}_1 \given M_1, \sigma_{M_1}^2\right) \\
    p(\hat{M}_2 \given M_2, \sigma_{M_2}) &=&
        \norm\left(\hat{M}_2 \given M_2, \sigma_{M_2}^2\right)
\end{eqnarray}

Our prior over the true 6D phase-space coordinates for each star, $p(\vec{w}_1,
\vec{w}_2)$, contains three components corresponding to the three scenarios:
\begin{enumerate}
    \item The stars are a bound, wide-binary system on a circular orbit;
    \item The stars are co-moving within some tolerance (velocity dispersion)
    but unbound;
    \item The stars are independently sampled from the field population.
\end{enumerate}
In case 1 (wide-binary), the prior can be expressed in terms of the 6D
barycentric phase-space coordinates of the pair, $\vec{w}_{\rm bary}$, and the
difference in 6D coordinates, $\Delta\vec{w}$, and is conditional on the true
masses of the stars and a (\emph{hack}) softening velocity dispersion, $V_1$,
introduced to approximately account for eccentricity:
\begin{eqnarray}
    p_1(\vec{w}_1, \vec{w}_2 \given M_1, M_2) &=&
        p(\vec{x}_{\rm bary} \given M_1, M_2) \,
        p(\Delta\vec{x} \given M_1, M_2) \, \\
    && \times \,
        p(\vec{v}_{\rm bary} \given M_1, M_2) \,
        p(\Delta\vec{v} \given M_1, M_2, V_1) \\
    p(\vec{x}_{\rm bary} \given M_1, M_2) &=& \mathcal{U}(-1,1)~[{\rm kpc}] \\
    p(\Delta\vec{x} \given M_1, M_2) &\propto& \left|\Delta\vec{x}\right|^{-3} \\
    p(\vec{v}_{\rm bary} \given M_1, M_2) &=&
      \mathcal{N}(\vec{v}_{\rm bary} \given 0, V_3) \\
    p(\Delta\vec{v} \given M_1, M_2, V_1) &=&
      \mathcal{N}\left(\Delta\vec{v} \given
        2\,\left[G \, (M_1+M_2)\right]/\left|\Delta\vec{x}\right|, V_1\right) \\
\end{eqnarray}
The peculiarity here is that for this component, the prior is only over the
magnitude of the coordinate separation and the magnitude of the velocity
difference (2 of 6 dimensions), the other 4 dimensions correspond to angles,
which we can assume to be isotropic.

Cases 2 and 3 are simple to write down.

%\bibliography{ref}



\end{document}
